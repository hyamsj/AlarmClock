
\newglossaryentry{Buildystem}
{
  name=Buildystem,
  description={Linux Mint 17, Java JDK 1.8.1, Intellij 2016.3.5, JavaFx Scene Builder 2.0}
} 

\newglossaryentry{Referenzsystem}
{
  name=Referenzsystem,
  description={Linux Mint 17, Java JRE 1.8.1, Derby10.13.1.1}
} 

\newglossaryentry{Programmstart}
{
  name=Programmstart,
  description={Prozess, welcher das Programm initialisiert}
} 


 
\newglossaryentry{Standarddatenbank}
{
  name=Standarddatenbank,
  description={Derby, MySQL, SQLite}
} 


\newglossaryentry{JPA Driver}
{
  name=JPA Driver,
  description={JPA Driver und JPA Module sind Opensource CMS (Content Management Systeme)}
} 

\newglossaryentry{Benutzer}
{
  name=Benutzer,
  description={Ein Benutzer ist ein Mensch, welcher unser Programm benutzt. Falls nicht anders erwähnt, muss dieser
  weder registriert noch angemeldet sein}
} 


\newglossaryentry{Event}
{
  name=Event,
  description={Ein Event ist ein Ereignis, welches zu einem bestimmten Zeitpunkt stattfindet}
} 

\newglossaryentry{Einzelevent}
{
  name=Einzelevent,
  description={Ein Einzelevent ist ein Event, welches nicht periodisch wiederkehrend stattfindet, also können auch
   ein Treffen, welches unregelmässig stattfindet zu einem Einzelevent werden.}
} 

\newglossaryentry{GUI}
{
  name=GUI,
  description={Graphical User Interface. Ein GUI ist eine graphische Benutzeroberfläche. Sie hat die Aufgabe, das
  Programm für den Benutzer bedienbar zu machen}
} 

\newglossaryentry{CLI Wikipedia}
{
  name=CLI,
  description={Command Line Interface. Das CLI ist die Konsole, wird oft auch Terminal gennant. Sie steuert eine
  Software mittels Textmodus. Je nach Betriebssystem wird die Kommandozeile von einer Shell ausgewertet und die
  entsprechende Funktion ausgeführt.
  - Wikipedia}
}

\newglossaryentry{Shell}
{
  name=Shell,
  description={In der Informatik bezeichnet man als Shell die Software, die den Benutzer mit dem Computer verbindet.
  Die Shell ermöglicht zum Beispiel, Kerneldienste zu nutzen und sich über Systemkomponenten zu informieren oder sie zu
  bedienen. Die Shell ist in der Regel ein Teil des Betriebssystems.
  - Wikipedia
  }
}


\newglossaryentry{API}
{
  name=API,
  description={Application Programming Interface. Ein API ist ein Programmteil, der von einem Softwaresystem anderen
  Programmen zur Anbindung an das System zur Verfügung gestellt wird
  - Wikipedia}
} 

\newglossaryentry{Notification}
{
  name=Notification,
  description={Eine Notification ist eine Aktion des Programms, die den Benutzer auf ein Event aufmerksam machen soll.}
} 

\newglossaryentry{Konfiguration}
{
  name=Konfiguration,
  description={Die Konfiguration kann ein Programm auf die Bedürfnisse des Nutzers anpassen. So kann man beispielsweise
   die Spracheinstellung konfigurieren.}
} 

\newglossaryentry{Filtern}
{
  name=Filtern,
  description={Filtern bedeutet, dass man gewisse Informationen nur darstellt, wenn diese eine oder mehrere bestimmte
  Charakteristiken aufweist. So ist es zum Beispiel möglich, Events nach Kategorien zu filtern, so dass man nur Events
  aus einer bestimmten Kategorie sehen kann.}
} 


\newglossaryentry{Kategorien}
{
  name=Kategorien,
  description={Eine Kategorie hilft die Events zu klassieren. Denkbare Kategoreien sind zum Beispiel: Arbeit, Freizeit, Persönlich,Familie,Wichtig,Sportverein... }
} 

\newglossaryentry{Android}
{
  name=Android,
  description={Android ist ein Betriebssystem für Mobile Geräte wie Smartphones, Tablets etc.
   Es wird von der von Google gegründeten Open Handset Alliance entwickelt}
}

\newglossaryentry{Notification Infrastruktur}
{
  name=Notification Infrastruktur,
  description={Einige Desktop Environments  wie Gnome oder KDE bieten eine eigene Notification Infrastruktur, diese
  erlaubt es, ``Pop-Ups'' durch Systemkomponenten darzustellen. KDE benutzt dies beispielsweise um auf einen niedrigen
  Batterieladezustand hinzuweisen. Auch andere Programme können diese Notification-Infrastruktur nutzen. }
} 

\newglossaryentry{Desktop Environment}
{
  name=Desktop Environment,
  description={Desktop Environment, ist eine graphische Benutzeroberfläche für das Betriebssystem. Vor allem unter
  Unixoiden Betriebssystemen, hat man eine grosse Auswahl an Desktop Environments (KDE, gnome, w3..)}
} 

\newglossaryentry{Cronjob}
{
  name=Cronjob,
  description={Cronjob ist ein Unix Dienst, welcher dazu dient zu einem bestimmten Zeitpnkt Ereignisse auszulösen.}
} 



\newglossaryentry{IFTTT}
{
  name=IFTTT,
  description={IFTTT (die Abkürzung von If This Then That, ausgesprochen „ift“ wie in „Gift“[1]) ist ein Dienstanbieter,
  der es Benutzern erlaubt, verschiedene Webanwendungen (zum Beispiel Facebook, Evernote, Dropbox usw.) mit einfachen
  bedingten Anweisungen zu verknüpfen.
   -Wikipedia}
} 





