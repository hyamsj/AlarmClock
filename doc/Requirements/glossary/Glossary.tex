
\newglossaryentry{Buildystem}
{
  name=Buildystem,
  description={Linux Mint 17,Java JDK 1.8.1, Intellij 2016.3.5, JavaFx Scene Builder 2.0}
} 

\newglossaryentry{Referenzsystem}
{
  name=Referenzsystem,
  description={Linux Mint 17,Java JRE 1.8.1,Derby10.13.1.1}
} 

\newglossaryentry{Programmstart}
{
  name=Programmstart,
  description={TODO}
} 


 
\newglossaryentry{Standarddatenbank}
{
  name=Standarddatenbank,
  description={Derby,MySQL,SQLite}
} 


\newglossaryentry{JPA Driver}
{
  name=JPA Driver,
  description={TODO}
} 

\newglossaryentry{Benutzer}
{
  name=Benutzer,
  description={TODO Ein Benutzer ist ein Mensch, welcher unser Programm benutzt, falls nicht anders erwähnt, muss dieser weder registriert noch angemeldet sein}
} 



\newglossaryentry{Registrierter Benutzer}
{
  name=Registrierter Benutzer,
  description={TODO}
} 

\newglossaryentry{Event}
{
  name=Event,
  description={TODO Ein Event ist ein Ereignis, welches zu einem bestimmten Zeitpunkt stattfindet}
} 

\newglossaryentry{Einzelevent}
{
  name=Einzelevent,
  description={TODO Ein Einzelevent ist ein Event, welches nicht periodisch wiederkehrend stattfindet, also können auch
   ein Treffen, welches unregelmässig stattfindet zu einem Einzelevent werden.}
} 

\newglossaryentry{GUI}
{
  name=GUI,
  description={TODO Wikipdeia}
} 

\newglossaryentry{CLI Wikipedia}
{
  name=CLI,
  description={TODO Wikipedia}
} 

\newglossaryentry{API Wikipedia}
{
  name=API,
  description={TODO Wikipedia}
} 

\newglossaryentry{Notification}
{
  name=Notification,
  description={TODO Eine Notification ist eine Aktion des Programms, die den Benutzer auf ein Event aufmerksam machen soll.}
} 

\newglossaryentry{Konfiguration}
{
  name=Konfiguration,
  description={Die Konfiguration kann ein Programm auf die Bedürfnisse des Nutzers anpassen. So kann man beispielsweise die Spracheinstellung konfigurieren.}
} 



\newglossaryentry{Filtern}
{
  name=Filtern,
  description={TODO}
} 

\newglossaryentry{Gruppieren}
{
  name=Gruppieren,
  description={TODO}
} 

\newglossaryentry{Kategorien}
{
  name=Kategorien,
  description={TODO }
} 

\newglossaryentry{Android}
{
  name=Android,
  description={TODO}
} 

\newglossaryentry{Notification Infrastruktur}
{
  name=Notification Infrastruktur,
  description={TODO}
} 

\newglossaryentry{Desktop Environment}
{
  name=Desktop Environment,
  description={TODO}
} 

\newglossaryentry{Cronjob}
{
  name=Cronjob,
  description={TODO}
} 
\newglossaryentry{test}
{
  name=Cronjob,
  description={TODO}
} 

\newglossaryentry{schtask}
{
  name=schtask,
  description={TODO}
} 

\newglossaryentry{IFTTT}
{
  name=IFTTT,
  description={IFTTT (die Abkürzung von If This Then That, ausgesprochen „ift“ wie in „Gift“[1]) ist ein Dienstanbieter, der es Benutzern erlaubt, verschiedene Webanwendungen (zum Beispiel Facebook, Evernote, Dropbox usw.) mit einfachen bedingten Anweisungen zu verknüpfen. Wikipedia}
} 





