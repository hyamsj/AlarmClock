
\newglossaryentry{Buildystem}
{
  name=Buildystem,
  description={Linux Mint 17, Java JDK 1.8.1, Intellij 2016.3.5, JavaFx Scene Builder 2.0}
} 

\newglossaryentry{Referenzsystem}
{
  name=Referenzsystem,
  description={Linux Mint 17, Java JRE 1.8.1, Derby10.13.1.1}
} 

\newglossaryentry{Programmstart}
{
  name=Programmstart,
  description={Prozess welcher das Programm initialisiert}
} 


 
\newglossaryentry{Standarddatenbank}
{
  name=Standarddatenbank,
  description={Derby, MySQL, SQLite}
} 


\newglossaryentry{JPA Driver}
{
  name=JPA Driver,
  description={JPA Driver und JPA Module sind Opensource CMS (Content Management Systeme)}
} 

\newglossaryentry{Benutzer}
{
  name=Benutzer,
  description={Ein Benutzer ist ein Mensch, welcher unser Programm benutzt. Falls nicht anders erwähnt, muss dieser weder registriert noch angemeldet sein}
} 


\newglossaryentry{Event}
{
  name=Event,
  description={Ein Event ist ein Ereignis, welches zu einem bestimmten Zeitpunkt stattfindet}
} 

\newglossaryentry{Einzelevent}
{
  name=Einzelevent,
  description={Ein Einzelevent ist ein Event, welches nicht periodisch wiederkehrend stattfindet, also können auch
   ein Treffen, welches unregelmässig stattfindet zu einem Einzelevent werden.}
} 

\newglossaryentry{GUI}
{
  name=GUI,
  description={Graphical User Interface. Ein GUI ist eine Graphische Benutzeroberfläche. Sie hat die Aufgabe das Programm für den Benutzer bedienbar zu machen}
} 

\newglossaryentry{CLI Wikipedia}
{
  name=CLI,
  description={Command Line Interface. Das CLI ist die Konsole, wird oft auch Terminal gennant. Sie steuert eine
  Software mittels Textmodus. Je nach Betriebssystem wird die Kommandozeile von einer Shell ausgewertet und die
  entsprechende Funktion ausgeführt.
  - Wikipedia}
}

\newglossaryentry{Shell}
{
  name=Shell,
  description={In der Informatik bezeichnet man als Shell die Software, die den Benutzer mit dem Computer verbindet.
  Die Shell ermöglicht zum Beispiel, Kerneldienste zu nutzen und sich über Systemkomponenten zu informieren oder sie zu
  bedienen. Die Shell ist in der Regel ein Teil des Betriebssystems.
  - Wikipedia
  }
}


\newglossaryentry{API Wikipedia}
{
  name=API,
  description={TODO Wikipedia}
} 

\newglossaryentry{Notification}
{
  name=Notification,
  description={TODO Eine Notification ist eine Aktion des Programms, die den Benutzer auf ein Event aufmerksam machen soll.}
} 

\newglossaryentry{Konfiguration}
{
  name=Konfiguration,
  description={Die Konfiguration kann ein Programm auf die Bedürfnisse des Nutzers anpassen. So kann man beispielsweise die Spracheinstellung konfigurieren.}
} 

\newglossaryentry{Filtern}
{
  name=Filtern,
  description={TODO}
} 

\newglossaryentry{Gruppieren}
{
  name=Gruppieren,
  description={TODO}
} 

\newglossaryentry{Kategorien}
{
  name=Kategorien,
  description={TODO }
} 

\newglossaryentry{Android}
{
  name=Android,
  description={TODO}
} 

\newglossaryentry{Notification Infrastruktur}
{
  name=Notification Infrastruktur,
  description={TODO}
} 

\newglossaryentry{Desktop Environment}
{
  name=Desktop Environment,
  description={TODO}
} 

\newglossaryentry{Cronjob}
{
  name=Cronjob,
  description={TODO}
} 
\newglossaryentry{test}
{
  name=Cronjob,
  description={TODO}
} 

\newglossaryentry{schtask}
{
  name=schtask,
  description={TODO}
} 

\newglossaryentry{IFTTT}
{
  name=IFTTT,
  description={IFTTT (die Abkürzung von If This Then That, ausgesprochen „ift“ wie in „Gift“[1]) ist ein Dienstanbieter, der es Benutzern erlaubt, verschiedene Webanwendungen (zum Beispiel Facebook, Evernote, Dropbox usw.) mit einfachen bedingten Anweisungen zu verknüpfen. Wikipedia}
} 





