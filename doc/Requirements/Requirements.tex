\documentclass[a4paper,10pt]{report}
\usepackage[utf8]{inputenc}
\usepackage[ngerman]{babel}

% Title Page
\title{Projekt 1}
\author{Jonathan Hyams \\Pascal Schmalz}


\begin{document}
\maketitle
\tableofcontents

\begin{abstract}
\end{abstract}
\section{Zweck des Dokument}
\section{Kurzbeschreibung}
Das Ziel des Projektes ist einen Ersatz zum Programm kAlarm zu entwickeln.
Das Program kAlarm erlaubt es den User Benutzerdefinierte Erinnerungen zu erstellen. Mittels Pop-Up Windows wird der User dann zur gegebenen Zeit daran erinnert.
Im gegensatz zu kAlarm soll das zu erstellende Produkt Platformübergreifend verfügbar sein. Wie kAlarm soll dieses Produkt unter einer Open Source Lizenz entwickelt werden.
\section{Projektziele}
Es soll ein Timer erstellt werden, welcher auf den 3 grossen Computerbetriebssystemen (Windows, OSX, Linux)  läuft.
Wiederkehrende Events sollen definiert werden können.
Es soll Eventkategorien geben.
Das Programm soll durch den Benuzer mittels  eines externen Configfile an seine Bedürfnisse angepasst werden können.
Das Programm soll Datenbankunabhänig sein. (Abklären mit dem Stakeholder!!)
\section {Systemabgrenzung}
\subsection{Prozessumfeld}
\subsection{Systemumfeld}
\subsection{Randbedingungen}
\section{Anforderungen}
\subsection{Quellen und Vorgehen}
\subsection{Technische Anforderungen}
\subsection{Qualitätsanforderungen}
\section{Glossar}
\listoffigures
\listoftables
%TODO bibography

\section{Anhang}

\subsection{Abstimmung der Anforderungen}
\subsection{Definition of Ready - Checklist}
\section{Versionskontrolle}
Version 0.0.1
%\immediate\write18{/usr/local/bin/my-shell-script.sh > scriptoutput.tex}
%git rev-list --count --first-parent HEAD
%9	Literaturverzeichnis	4







\end{document}
