%\documentclass[a4paper,10pt]{report}
\documentclass[11pt,titelpage]{scrreprt}
\usepackage[utf8]{inputenc}
\usepackage[ngerman]{babel}
\usepackage{graphicx}
\usepackage{fancyhdr}


% Title Page
\title{Alarm Clock }
\author{Jonathan Hyams \\Pascal Schmalz}
\titlehead{\centering\includegraphics[width=6cm]{img/clock.png}}

%Make the Header
\makeatletter
\let\runauthor\@author
\let\runtitle\@title
\makeatother
\rhead{\runauthor}
\chead{\runtitle}
\lhead{\begin{picture}(0,0) \put(0,0){\includegraphics[scale=0.5]{img/bfh.png}} \end{picture}}



\begin{document}

\thispagestyle{empty}
\maketitle
\tableofcontents

\pagestyle{fancy}


\begin{abstract}
\end{abstract}
\section{Zweck des Dokument}
TODO this is new text

\section{Kurzbeschreibung}
Das Ziel des Projektes ist einen Ersatz zum Programm kAlarm zu entwickeln.
Das Program kAlarm erlaubt es den User Benutzerdefinierte Erinnerungen zu erstellen. Mittels Pop-Up Windows wird der User dann zur gegebenen Zeit daran erinnert.
Im gegensatz zu kAlarm soll das zu erstellende Produkt Platformübergreifend verfügbar sein. Wie kAlarm soll dieses Produkt unter einer Open Source Lizenz entwickelt werden.
\section{Projektziele}
Es soll ein Timer erstellt werden, welcher auf den 3 grossen Computerbetriebssystemen (Windows, OSX, Linux)  läuft.
Wiederkehrende Events sollen definiert werden können.
Es soll Eventkategorien geben.
Das Programm soll durch den Benuzer mittels  eines externen Configfile an seine Bedürfnisse angepasst werden können.
Das Programm soll Datenbankunabhänig sein. ( Abklären mit dem Stakeholder!!)
\section{Stakeholders}
\begin{itemize}
\item{Auftragsgeber: Prof. Claude Furrer}
\item{Auftragsnehmer: Jonathan Hyams, Pascal Schmalz}
\item{Benutzer: FOSS Communitiy}
\end{itemize}
\section {Systemabgrenzung}
\subsection{Geschäftsprozesse}
Das Abspeichern von Remindern. Unsere Lösung lässt sich in beliebig viele Prozesse integrieren.
\subsection{Systeme}
Unser System wirkt mit Datebanken zusammen. Es baut auf Betriebsystemskomponenten auf.
Je nach gewünschter Notification möglichkeiten kann mit E-Mails, SMS und Systemnotifications gearbeitet werden.
\subsection{Randbedingungen}
Es muss Betriebssystem und Datenbank agnostisch sein. Das Projekt muss unter einer Opensource Lizenz laufen.


\subsection{Prozessumfeld}
\subsection{Systemumfeld}
\subsection{Randbedingungen}
\section{Anforderungen}
\subsection{Quellen und Vorgehen}
\subsection{Technische Anforderungen}
\subsection{Qualitätsanforderungen}
\section{Glossar}
\listoffigures
\listoftables
%TODO bibography

\section{Anhang}

\subsection{Abstimmung der Anforderungen}
\subsection{Definition of Ready - Checklist}
\section{Versionskontrolle}
Manuelle Version: 0.0.1
\\

\noindent
Automatische Versionierung:
%\immediate\write18{../script/printGitVersionNumber.sh}
%\input{automaticGitVersionNumber}
%git rev-list --count --first-parent HEAD
%9	Literaturverzeichnis	4

\immediate\write18{../script/versionInfo.sh}
Fetching version information failed. Please enable shell-escape in your \LaTeX \~  compiler.

\immediate\write18{../script/cleanup.sh}
%\immediate\write18{../script/clean.sh}







\end{document}
