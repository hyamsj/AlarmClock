\subsection{Class: Model}
Die Model Klasse ist für die Daten Zuständig. Sie hat eine Liste mit allen Remindern als Klassenvariable. Die Model Klasse ladet die gespeicherten Daten über den DataBaseAdapter.
Dies macht sie im Konstruktor. Die Methode addReminder(), getReminders(), removeReminder() und removeReminders() sind relativ selbsterklärend.

Die bindData() Methode ist ein bisschen interessanter. Da die reminders Liste eine ObservableList ist, kann man mit ihr ein Binding erstellen. Wir haben es so aufgebaut, dass sobald
sich irgendetwas in der Liste ändert, die Methode adapter.save(reminders) aufgerufen wird. So gehen die Reminders nicht verloren.

\begin{lstlisting}
reminders.addListener(Poller.getInstance()::onChanged);
\end{lstlisting}
Dieses Statement macht zwei Sachen. Poller.getInstance gibt einen Poller zurück. Das onChanged macht, dass wenn sic etwas in der Reminder Liste ändert, der Poller notifiziert wird.

Die redo() und undo() Methoden kann man die Letzte Aktion in der Liste rückgängig machen. Das GUI selbst hat diese Funktionalität noch nicht, aber die Methoden machen auch wirklich das, was sie sollen.


\subsection{Interface: DateBaseAdapter}
Das DataBaseAdapter Interface stellt 2 Methoden zur Verfügung, die save() und die load() Methode. Save verlangt als Parameter die Liste von Reminders die Abggespeichert werden
sollen.

\subsection{Class: BinaryDBAdapter}
Die BinaryDBAdapter Klasse implementiert das DataBaseAdapter Interface und Serialisiert die Binäre Datenbank unter dem Namen reminders.ser. Die load() Methode schaut zuerst ob die DB
überhaupt existiert. Wenn nicht, wird eine neue erstellt. Dies machen wir mittels einem ObjectInputStream. Das Object das gespeichert wird (die Liste der Reminders),
casten wir zu einer ArrayList, und diese ArrayListe wird dann in eine ReminderList gefüllt.
Die save() Methode wird immer aufgerufen, sobald sich irgendetwas in der Tabelle vom GUI ändert. Das Abspeichern machen wir mittels einem ObjectOutputStream.
