\subsection{Class: Controller}
Die Controller Klasse arbeit eng mit der mainWindow.fxml Datei zusammen. Würde man dieser Klasse den Namen ändern, so müsste man im fxml das Statement
\begin{lstlisting}
fx:controller="alarmClock.controller.Controller"
\end{lstlisting}
anpassen

In der fxml Datei hat jedes Item im GUI eine ID. Damit man im Java Code auf die jeweiligen Komponenten zugreifen kann, hängt man bei der Initialisierung ein @FXML Tag dran, damit das Programm weiss, mit welchen Komponenten ehr umgeht.
\begin{lstlisting}
    @FXML
    private TextField subjectField;
\end{lstlisting}

\begin{lstlisting}
  <TextField fx:id="subjectField" prefHeight="25.0" prefWidth="107.0" promptText="Subject"\">TODO</TextField>
\end{lstlisting}

Da in beiden Dateien subjectField gleich heisst, kann man nun problemlos im Java Code auf das gewünschte Feld zugreifen.
Man kann auch definieren, welche Methode aufgerufen werden soll, wenn man bsp einen Button drückt. Für addButtonPressed() und rmButtonPressed() haben wir das gemacht.

\begin{lstlisting}
onAction="#addButtonPressed"
\end{lstlisting}

Im tag des AddButtons (in der fxml Datei) nennen wir die Methode die aufgerufen werden soll "addButtonPressed". Damit der Java-Compiler merkt, dass er eine FXML Kompenente suchen muss, fügen wir vor der Methode noch ein @FXML tag hinzu.
\begin{lstlisting}
    @FXML
        public void addButtonPressed() {
        ...
        }
\end{lstlisting}

Die addButtonPressed() Methode fügt Reminders in die Tabelle hinzu, solange der Input legal ist. Das heisst die Felder dürfen nicht leer sein (getValue != null). Wenn der Button gedrückt wird, werden die Felder wieder leer gemacht.


Die rmButtonPressed() Methode löscht Reminders die man mit der Maus selektiert hat.

\begin{lstlisting}
    ReminderList reminderSelected;
    reminderSelected = new ReminderList(reminderTable.getSelectionModel().getSelectedItems());
    model.removeReminders(reminderSelected);
\end{lstlisting}

Wir kreieren eine Liste und fügen alle selektierten Reminders hinzu, und löschen diese dann aus der Tabelle.


Die Methode Initialize haben wir reingenommen, da JavaFX Komponenten erst nach dem Ausführen des Konstrukters erstellt werden. Sie dient als Konstruktor.
\begin{lstlisting}
  BooleanBinding addBinding = subjectField.textProperty().isNotEmpty().and(datePickerField.valueProperty().isNotNull());
  addButton.disableProperty().bind(addBinding.not());
\end{lstlisting}
Das Binding oben deaktiviert den addButton wenn der Input nicht valid ist.

\begin{lstlisting}
reminderTable.getSelectionModel().setSelectionMode(
                SelectionMode.MULTIPLE
        );
\end{lstlisting}
Dies erlaubt dem User mehrere Items in der Tabelle zu markieren.

Um die Initialisierung des Models mussten wir noch ein try/ catch Block hinzufügen, da das Model Exceptions werfen kann.

\begin{lstlisting}
reminderTable.setItems(model.getReminders());
\end{lstlisting}
Fügt beim Starten des Programms Reminders in die Tabelle, die in der DB gespeichert wurden.
