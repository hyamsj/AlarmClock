%\documentclass[a4paper,10pt]{report}
\documentclass[11pt,titelpage]{scrartcl}
\usepackage[utf8]{inputenc}
\usepackage[ngerman]{babel}
\usepackage{graphicx}
\usepackage{fancyhdr}
\usepackage{fancyref}
\usepackage{hyperref}
\usepackage{lscape}
\usepackage{color}
\definecolor{javared}{rgb}{0.6,0,0} % for strings
\definecolor{javagreen}{rgb}{0.25,0.5,0.35} % comments
\definecolor{javapurple}{rgb}{0.5,0,0.35} % keywords
\definecolor{javadocblue}{rgb}{0.25,0.35,0.75} % javadoc
\definecolor{javared}{rgb}{0.6,0,0} % for strings
\definecolor{lightgrey}{rgb}{0.97,0.97,0.97}
\definecolor{grey}{rgb}{0.3,0.3,0.3}
\definecolor{darkgreen}{rgb}{0,0.6,0}

\usepackage{listings}
\lstset{
language=java,
keywordstyle=\color{javapurple}\bfseries,
stringstyle=\color{javared},
commentstyle=\color{javagreen},
morecomment=[s][\color{javadocblue}]{/**}{*/},
numberstyle=\tiny\color{grey},
numberfirstline=true,
firstnumber=1,
stepnumber=5,
numbers=left,
numbersep=10pt,
tabsize=4,
breaklines=true,
showspaces=false,
showstringspaces=false,
backgroundcolor=\color{lightgrey}
}


%must be before gloassary stuff\usepackage{hyperref}
\usepackage[toc]{glossaries}

%\makeglossaries
%
\newglossaryentry{Buildystem}
{
  name=Buildystem,
  description={Linux Mint 17, Java JDK 1.8.1, Intellij 2016.3.5, JavaFx Scene Builder 2.0}
} 

\newglossaryentry{Referenzsystem}
{
  name=Referenzsystem,
  description={Linux Mint 17, Java JRE 1.8.1, Derby10.13.1.1}
} 

\newglossaryentry{Programmstart}
{
  name=Programmstart,
  description={Prozess welcher das Programm initialisiert}
} 


 
\newglossaryentry{Standarddatenbank}
{
  name=Standarddatenbank,
  description={Derby, MySQL, SQLite}
} 


\newglossaryentry{JPA Driver}
{
  name=JPA Driver,
  description={JPA Driver und JPA Module sind Opensource CMS (Content Management Systeme)}
} 

\newglossaryentry{Benutzer}
{
  name=Benutzer,
  description={Ein Benutzer ist ein Mensch, welcher unser Programm benutzt. Falls nicht anders erwähnt, muss dieser
  weder registriert noch angemeldet sein}
} 


\newglossaryentry{Event}
{
  name=Event,
  description={Ein Event ist ein Ereignis, welches zu einem bestimmten Zeitpunkt stattfindet}
} 

\newglossaryentry{Einzelevent}
{
  name=Einzelevent,
  description={Ein Einzelevent ist ein Event, welches nicht periodisch wiederkehrend stattfindet, also können auch
   ein Treffen, welches unregelmässig stattfindet zu einem Einzelevent werden.}
} 

\newglossaryentry{GUI}
{
  name=GUI,
  description={Graphical User Interface. Ein GUI ist eine Graphische Benutzeroberfläche. Sie hat die Aufgabe das
  Programm für den Benutzer bedienbar zu machen}
} 

\newglossaryentry{CLI Wikipedia}
{
  name=CLI,
  description={Command Line Interface. Das CLI ist die Konsole, wird oft auch Terminal gennant. Sie steuert eine
  Software mittels Textmodus. Je nach Betriebssystem wird die Kommandozeile von einer Shell ausgewertet und die
  entsprechende Funktion ausgeführt.
  - Wikipedia}
}

\newglossaryentry{Shell}
{
  name=Shell,
  description={In der Informatik bezeichnet man als Shell die Software, die den Benutzer mit dem Computer verbindet.
  Die Shell ermöglicht zum Beispiel, Kerneldienste zu nutzen und sich über Systemkomponenten zu informieren oder sie zu
  bedienen. Die Shell ist in der Regel ein Teil des Betriebssystems.
  - Wikipedia
  }
}


\newglossaryentry{API}
{
  name=API,
  description={Application Programming Interface. Ein API ist ein Programmteil, der von einem Softwaresystem anderen
  Programmen zur Anbindung an das System zur Verfügung gestellt wird
  - Wikipedia}
} 

\newglossaryentry{Notification}
{
  name=Notification,
  description={Eine Notification ist eine Aktion des Programms, die den Benutzer auf ein Event aufmerksam machen soll.}
} 

\newglossaryentry{Konfiguration}
{
  name=Konfiguration,
  description={Die Konfiguration kann ein Programm auf die Bedürfnisse des Nutzers anpassen. So kann man beispielsweise
   die Spracheinstellung konfigurieren.}
} 

\newglossaryentry{Filtern}
{
  name=Filtern,
  description={TODO}
} 

\newglossaryentry{Gruppieren}
{
  name=Gruppieren,
  description={TODO}
} 

\newglossaryentry{Kategorien}
{
  name=Kategorien,
  description={TODO }
} 

\newglossaryentry{Android}
{
  name=Android,
  description={Android ist ein Betriebssystem für Mobile Geräte wie Smartphones, Tablets etc.
   Es wird von der von Google gegründeten Open Handset Alliance entwickelt}
}

\newglossaryentry{Notification Infrastruktur}
{
  name=Notification Infrastruktur,
  description={TODO}
} 

\newglossaryentry{Desktop Environment}
{
  name=Desktop Environment,
  description={Deskto Environment, ist eine Graphsche Benutzeroberfläche für das Betriebssystem. Vorallem unter
  Unixoiden Betriebssystemen, hat man eine grosse Auswahl an Desktop Environments (KDE, gnome, w3..)}
} 

\newglossaryentry{Cronjob}
{
  name=Cronjob,
  description={Cronjob ist ein Unix Dienst, welcher dazu dient zu einem bestimmten Zeitpnkt Ereignisse auszulösen.}
} 



\newglossaryentry{IFTTT}
{
  name=IFTTT,
  description={IFTTT (die Abkürzung von If This Then That, ausgesprochen „ift“ wie in „Gift“[1]) ist ein Dienstanbieter,
  der es Benutzern erlaubt, verschiedene Webanwendungen (zum Beispiel Facebook, Evernote, Dropbox usw.) mit einfachen
  bedingten Anweisungen zu verknüpfen.
   -Wikipedia}
} 








% Title Page
\title{Alarm Clock:\\ Technische Dokumentation }
\author{Jonathan Hyams \\Pascal Schmalz}
\titlehead{\centering\includegraphics[width=6cm]{../Requirements/img/clock.png}}

%Make the Header
\makeatletter
\let\runauthor\@author
\let\runtitle\@title
\makeatother
\rhead{\runauthor}
\chead{\runtitle}
%\lhead{\begin{picture}(0,0) \put(0,0){\includegraphics[scale=0.5]{img/bfh.png}} \end{picture}}


\begin{document}

\thispagestyle{empty}
\maketitle
\pagebreak
\tableofcontents

\pagestyle{fancy}


\begin{abstract}
\end{abstract}
\pagebreak

\section{Zweck des Dokument}
Dieses Dokument dient für die Leute, die unseren Code analysieren, verstehen und bewerten wollen. Wir erklären hier, wie unser Programm strukturiert ist, wie es funktioniert
und warum wir einige Entscheidungen getroffen haben.
\section{Architektur}
Unser Programm implementiert ein MVC Pattern. Die Komponenten Model und Controller sind als Java Pakete schnell ersichtlich. Die View wird aber nicht in einem Java-Packet ausprogrmamiert. Da wir JavaFx nutzen, wird die View durch JavaFx definiert. Die Implementierungsdetails findet man unter resources/mainWindow.fxml. Auch das Notification Paket könnte man zur View zählen.

Die wichtigsten Komponenten sieht man in der Abbildung Systemübersicht.
\begin{landscape}
\begin{figure}
  \centering
    \includegraphics[width=1\textwidth]{../uml/uebersicht01.png}
  \caption{Systemübersicht}
  \label{fig:overview}
\end{figure}
\end{landscape}

Spezielle Aufmerksamkeit bedarf das Auslösen eines Reminders.
\subsection{Auslösen Notification}

Der Poller stösst regelmässig den NotificationHandler an.
In diesem werden Kriterien definiert, nach welchen die Reminders gefiltert werden, zum Beispiel alle Reminders, welche in der nächsten Stunde beginnen.



Dann wird über alle Reminders iteriert. Jedem Reminder, wird dabei der Filter an die NotifyIf() Funktion übergeben.
Der Reminder testet nun anhand des Filters selbstständig, ob das Kriterium zutrifft oder nicht. Dazu nutzt er seine meetsCriteria() Funktion.
Falls diese eine positive Antwort gibt, wird die doNotify() Funktion aufgerufen. Diese lädt eine Liste mit den Notifications, welche im ConfigReader definiert wird.
Jeder dieser Notifications wird dem Reminder übergeben, welcher die Notifications auslösen will. Dann wird die Notifications abgesendet.

\subsection{Konfiguration}
Die Konfigurationsmöglichkeiten sind im ConfigReader File zentral gelöst. Somit wird es später auch leicht möglich, über diese Klasse ein Konfigurationsfile
einzulesen, und die entsprechenden Komponenten zu konfigurieren.

\section{Main}
\subsection{Class: Main}
Dies ist die Klasse die die Main methode enthält. Sie erbt von der Klasse Application, da Sie das JavaFX-GUI launched.
Application implementiert die start methode, die ein Stage als Parameter hat. Diese Stage ist das Hauptfenster mit der Tabelle von Reminders.

\begin{lstlisting}
  Platform.setImplicitExit(false);
\end{lstlisting}
Wenn Platform.setImplicitExit true ist, werden Befehle wie Platform.runlater() ignoriert, sobald alle JavaFX Fenster geschlossen sind.
Das würde heissen,das wenn das Hauptfenster geschlossen wurde (und alle Popups), würden spätere Popups nicht mehr erscheinen.
Dies wollen wir verhindern, also ist es auf false gesetzt.

\begin{lstlisting}
   Parent root = FXMLLoader.load(getClass().getResource(windowName));
\end{lstlisting}
windowName ist der Name der fxml Datei (mainWindow.fxml), in der das Aussehen und der Controller definiert wird. "mainWindow.fxml" befindet sich im package "resources".


\begin{lstlisting}
    if (new ConfigReader().isEnableDarkMode()) {
            scene.getStylesheets().add("dark.css");
        } else {
            scene.getStylesheets().add("styles.css");
        }
\end{lstlisting}
Der ConfigReader kann so angepasst werden, das man hier das gewünsche CSS File bekommt.

\section{Controller}
\subsection{Class: Controller}
Die Controller Klasse arbeit eng mit der mainWindow.fxml Datei zusammen. Würde man dieser Klasse den Namen ändern, so müsste man im fxml das Statement
\begin{lstlisting}
fx:controller="alarmClock.controller.Controller"
\end{lstlisting}
anpassen

In der fxml Datei hat jedes Item im GUI eine ID. Damit man im Java Code auf die jeweiligen Komponenten zugreifen kann, hängt man bei der Initialisierung ein @FXML Tag dran, damit das Programm weiss, mit welchen Komponenten ehr umgeht.
\begin{lstlisting}
    @FXML
    private TextField subjectField;
\end{lstlisting}

\begin{lstlisting}
  <TextField fx:id="subjectField" prefHeight="25.0" prefWidth="107.0" promptText="Subject"\">TODO</TextField>
\end{lstlisting}

Da in beiden Dateien subjectField gleich heisst, kann man nun problemlos im Java Code auf das gewünschte Feld zugreifen.
Man kann auch definieren, welche Methode aufgerufen werden soll, wenn man bsp einen Button drückt. Für addButtonPressed() und rmButtonPressed() haben wir das gemacht.

\begin{lstlisting}
onAction="#addButtonPressed"
\end{lstlisting}

Im tag des AddButtons (in der fxml Datei) nennen wir die Methode die aufgerufen werden soll "addButtonPressed". Damit der Java-Compiler merkt, dass er eine FXML Kompenente suchen muss, fügen wir vor der Methode noch ein @FXML tag hinzu.
\begin{lstlisting}
    @FXML
        public void addButtonPressed() {
        ...
        }
\end{lstlisting}

Die addButtonPressed() Methode fügt Reminders in die Tabelle hinzu, solange der Input legal ist. Das heisst die Felder dürfen nicht leer sein (getValue != null). Wenn der Button gedrückt wird, werden die Felder wieder leer gemacht.


Die rmButtonPressed() Methode löscht Reminders die man mit der Maus selektiert hat.

\begin{lstlisting}
    ReminderList reminderSelected;
    reminderSelected = new ReminderList(reminderTable.getSelectionModel().getSelectedItems());
    model.removeReminders(reminderSelected);
\end{lstlisting}

Wir kreieren eine Liste und fügen alle selektierten Reminders hinzu, und löschen diese dann aus der Tabelle.


Die Methode Initialize haben wir reingenommen, da JavaFX Komponenten erst nach dem Ausführen des Konstrukters erstellt werden. Sie dient als Konstruktor.
\begin{lstlisting}
  BooleanBinding addBinding = subjectField.textProperty().isNotEmpty().and(datePickerField.valueProperty().isNotNull());
  addButton.disableProperty().bind(addBinding.not());
\end{lstlisting}
Das Binding oben deaktiviert den addButton wenn der Input nicht valid ist.

\begin{lstlisting}
reminderTable.getSelectionModel().setSelectionMode(
                SelectionMode.MULTIPLE
        );
\end{lstlisting}
Dies erlaubt dem User mehrere Items in der Tabelle zu markieren.

Um die Initialisierung des Models mussten wir noch ein try/ catch Block hinzufügen, da das Model Exceptions werfen kann.

\begin{lstlisting}
reminderTable.setItems(model.getReminders());
\end{lstlisting}
Fügt beim Starten des Programms Reminders in die Tabelle, die in der DB gespeichert wurden.

\section{Model}
\subsection{Class: Model}
Die Model Klasse ist für die Daten zuständig. Sie hat eine Liste mit allen Remindern als Klassenvariable. Die Model Klasse ladet die gespeicherten Daten über den DataBaseAdapter.
Dies macht sie im Konstruktor. Die Methoden addReminder(), getReminders(), removeReminder() und removeReminders() sind relativ selbsterklärend.

Die bindData() Methode ist ein bisschen interessanter. Da die Reminderliste eine ObservableList ist, kann man mit ihr ein Binding erstellen. Wir haben es so aufgebaut, dass, sobald
sich irgend etwas in der Liste ändert, die Methode adapter.save(reminders) aufgerufen wird. So gehen die Reminders nicht verloren.

\begin{lstlisting}
reminders.addListener(Poller.getInstance()::onChanged);
\end{lstlisting}
Dieses Statement macht zwei Sachen. Poller.getInstance gibt einen Poller zurück. Das onChanged macht, dass, wenn sich etwas in der Reminder Liste ändert, der Poller notifiziert wird.

Die redo() und undo() Methoden kann man die letzte Aktion in der Liste rückgängig machen. Das GUI selbst hat diese Funktionalität noch nicht, aber die Methoden machen auch wirklich
das, was sie sollen.


\subsection{Interface: DateBaseAdapter}
Das DataBaseAdapter Interface stellt 2 Methoden zur Verfügung, die save() und die load() Methode. Save verlangt als Parameter die Liste von Reminders die abgespeichert werden
sollen.

\subsection{Class: BinaryDBAdapter}
Die BinaryDBAdapter Klasse implementiert das DataBaseAdapter Interface und Serialisiert die Binäre Datenbank unter dem Namen reminders.ser. Die load() Methode schaut zuerst, ob die DB
überhaupt existiert. Wenn nicht, wird eine neue erstellt. Dies machen wir mittels einem ObjectInputStream. Das Objekt das gespeichert wird (die Liste der Reminders),
casten wir zu einer ArrayList, und diese ArrayListe wird dann in eine ReminderList gefüllt.
Die save() Methode wird immer aufgerufen, sobald sich irgend etwas in der Tabelle vom GUI ändert. Das Abspeichern machen wir mittels einem ObjectOutputStream.

\subsection{Class: Reminder}
Der Reminder ist das Herzstück unseres Programms. Alles was in der Tabelle ist, ist ein Reminder. Ursprünglich waren die Klassenvariablen SimpleStringProperties, da die Tabelle
solche verlangt. Das Problem war aber, dass sich diese nicht serialisieren lassen. Also haben wir die Klasse ein wenig anders strukturiert. Anstatt SimpleStringProperties
direkt in der Klasse zu Speichern, werden diese einfach von den Methoden zurückgegeben.
Das heisst anstatt wie folgt:
\begin{lstlisting}
private SimpleStringProperty subject = new SimpleStringProperty();

public SimpleStringProperty getSubjectProperty() {
  return subject;
}
\end{lstlisting}

Machen wir es so:
\begin{lstlisting}
private String subject = "";

public SimpleStringProperty getSubjectProperty() {
  return new SimpleStringProperty(subject);
}
\end{lstlisting}
So hat man keine Probleme mehr mit dem Serialisieren.

\subsubsection{Class: notifyIf}
Wie bereits in der Sektion  Architektur erwähnt, nimmt die notifyIf() Funktion einen einzelnen CriteriaTester oder eine  Liste von  CriteriaTester entgegen. Falls alle Kriterien,
welche darin definiert wurden zutreffen, wird die this.doNotify aufgerufen. Als boolscher Wert wird zurückgegeben, ob dies der Fall ist oder nicht.
\subsubsection{Class: meetsCriteria}
Die meetsCritera() Funktion funktioniert wie die notifyIf() Funktion. Mit dem Unterschied, dass this.doNotify() nicht aufgerufen wird. Sie gibt lediglich einen boolschen Wert als
Antwort zurück.

\subsubsection{Class: doNotify}
Falls doNotify() aufgerufen wird, iteriert der Reminder durch alle vorkonfigurierten Notifications und lässt jede mit der Funktion Notification.send() eine Notification
absenden. Somit ist es möglich mehrere verschiedene Notifications absenden zu lassen.


\subsection{Class: ReminderList}
Die ReminderList ist aus zwei Gründen entstanden.
Wir hatten sehr häufig eine beobachtbare (Observable) Liste von Remindern als Inputparameter in Funktionen zu definieren. Dies verschlechtert die Leserlichkeit, welche beim Code
sehr wichtig ist. Durch das definieren einer eigenen Klasse, kann man nun das besser lesbare ReminderList schreiben.

Ausserdem bietet uns die ReminderList auch die Möglichkeit, den Zustand der ReminderList in einer History zu speichern, wenn man diese ändert.
Somit kann man Änderungen an der Liste später rückgängig machen. Durch das Hinzufügen einer undoneHistory, kann man nun auf der Zeitachse in beide Richtungen navigieren.

Leider spielte das undo/ redo nicht gut mit JavaFX zusammen, so dass Änderungen gar nicht im GUI übernommen wurden. Die entsprechenden Buttons im GUI haben wir deshalb wieder entfernt.
Die Undo/Redo Funktionalität könnte aber in einem Nachfolgeprojekt interessant werden. Deshalb haben wir diese nicht gelöscht.

\subsection{Class: Specific Reminder}
Dies sind eine Spezialiesierung der Superklasse Reminder. Der einzige Unterschied besteht darin, dass ``SpecificReminder'' noch tags speichern. Diese tags können dann mit den
Filtern gefiltert werden.

\subsection{Class: Poller}
Der Poller folgt dem Singleton Pattern. Er läuft in einem eigenen Thread und ruft regelmässig (eingestellt ist jede Sekunde) den NotificationHandler auf.
Der Poller kann als Observer auf die ReminderList ``aufpassen'', somit kann er jede Änderung, welche vorgenommen wird sofort verarbeiten und gegebenenfalls via NotificationHandler
eine passende Notification absenden. Dies wird vor allem dann wichtig, wenn die Polling Zeit auf einen sehr grossen Wert gesetzt wird.

\subsection{Class: ConfigReader}
Der ConfigReader müsste eigentlich von einer Config Datei lesen, die der User anpassen kann. Dazu hat uns leider die Zeit nicht mehr gereicht.
Im ConfigReader kann man einstellen ob man
\begin{itemize}
  \item JavaFX Notifications überhaupt erhalten will
  \item Notifications auf der Konsole haben will
  \item ein Dark Mode haben will
  \item vergangene Reminders sehen will
  \item Reminder sehen will, die in der nächsten Sekunde eintreffen
  \item Reminder sehen will, die im nächsten Monat eintreffen
\end{itemize}

\subsubsection{Konfiguration}
Die Konfigurationsmöglichkeiten sind im ConfigReader File zentral gelöst. Damit wird es später auch leicht möglich, über diese Klasse ein Konfigurationsfile einzulesen und
die entsprechenden Komponenten zu konfigurieren. Solange dies nicht über ein Externes File eingelesen wird, wird die Konfiguration direkt in den Boolschen Werten gesetzt.

Bei den NotificationTypes muss man ein Objekt der gewünschten Notification eingeben, um einen neuen NotificationType zu konfigurieren. Dies hat den Vorteil einer Typesicherheit.
Ausserdem kann somit der Reminder im Reminder.doNotify() einfach über die verschiedenen Notifications iterieren, welcher er benutzen soll.

\section{Filtering}
\subsection{ Higher Order Functions}
  Die Marketingabteilung von Oracle behauptet gerne Java seie auch Funktional. Higer Order Functions werden aber nicht wirklich unterstützt. 
  TODO wikipedia Higer Order function. 
  Funktionen als Return Values oo
  
  Java unterstüzt leider keine Higher order functions. Man kann also keine Funktion als Inputparameter übergeben.
  Mittels Lambdas ist es lediglich möglich, die eine Funktion ausführen zu lassen und den Rückgabewert als Inputparameter weiter zu verwenden. Dies erlaubt eine kompaktere Notation. Dies reicht uns aber nicht, da wir den Remindern eine Funktion übergeben möchten, mit welcher jeder Reminder selber testet ob er eine Notification absenden soll. 
  
  \subsection{echte Higher Order Functions in Java}
  Um dies zu erreichen haben wir eine Form von Higher Order Functions mit Objekten nachgebaut. 
  Ein CriteraTester ist ein Objekt, welches als Wrapper für eine Funtion dient. Anstelle dieser Funktion übergibt man nun diesen FunktionsWrapper als Inputparameter. Somit konnten wir Funktionen als Inputparameter mittels Objektorientierten prinzipien nachbauen.
  Man muss nun für jede Funktion ein Objekt erstellen, welches das Interface CriteraTester implementiert.Und die Filterfunktion isTrue implementieren.
  Funktionen als Rückgabewert kann man so aber noch nicht wirklich nachbauen. Für uns war das aber nicht nötig.
  
  Dank den oben erwähnten Lambdas kann man dies auch elegan on the Fly erledigen. Da es aber vorkommen kann dass man einen Filter mehrmals benutzt, habe wir uns entschieden die Filter jeweils als eigene Klassen zu implementieren.
  \subsection{Code Beispiel}
  
       \begin{lstlisting}
       Reminder r;
       Collection<CriteriaTester> criteria = new ArrayList<>();
        criteria.add(new IsPassed());
         //     example how CriteraTester can be written on the fly
        //pus this to documentation
        criteria.add(
                r -> (!r.getTags().contains("hidden"))
        );
        //This lets the Reminder send a notification if the Reminder meets the criterias 
        //The first criteia it must pass it th IsPassed()
        //the second criteria is defined on the fly on  line number TODO x. it tests if it contains the tag "hidden"
        
     
        r.notifyIf(criteria);

       \end{lstlisting}
\section{Notification}
\subsection{Interface: Notification}
Das NotificationInterface gibt zwei Methoden vor:

\begin{lstlisting}
   void setReminder(Reminder reminder);
\end{lstlisting}
setReminder übergibt den Reminder, für den man eine Notifcation erstellen will.

\begin{lstlisting}
   void send();
\end{lstlisting}
send() wird vom Reminder aufgerufen, und zeigt dem User die Notification. Es gibt verschiene Arten von Notifications.
JavaFxNotification ist das Popup das den Reminder aufzeigt, der gerade aktuell ist. MultireminderNotification
zeigt alle schon vergangenen Reminders, und ConsoleReminder zeigt auf der Konsole (Shell) einen Reminder.
All diese Notification-Klassen implemenetieren Notification.


\subsection{Class: ConsoleNotification}
ConsoleNotification hat einen leeren Konstruktor aus mehreren Gründen. Wir haben ihn ähnlich wie den JavaFxNotification aufebaut,
und JavaFX verlangt einen leeren Konstuktor, also haben wir ihn hier beibehalten. Auch haben wir einen Konstruktor mit einem Reminder im Parameter gemacht,
also war der DefaultKonstruktor überschrieben.
Der ConfigReader benutzt diese Klasse auch (inklusive dem leeren Konstruktor) und benötigt keinen direkten Reminder, also haben wir ihn so stehen lassen.

Der Konstruktor mit dem Reminder im Parameter und die setReminderMethode machen eingentlich genau das gleiche. setReminders wird noch vom Notication Interface verlangt.
Die send() Methode wird vom Reminder aufgerufen wenn die Zeit soweit ist. Sie druckt einfach die toString() Methode des Reminders auf die Konsole.


\subsection{Class: JavaFxNotification}
Die JavaFxNotification Klasse ist identisch mit der ConsoleNotification, nur die send() Methode ist anders.
Die send() Methode enthält die statische Methode:
\begin{lstlisting}
  Platform.runLater( () -> {...});
\end{lstlisting}
In den geschweiften Klammern wird ein Popup Fenster erstellt. Da wir mit Threads arbeiten, und diese Popups verspätet aufgerufen werden,
müssen wir das GUI in das Platform.runlater einpacken. Die JavaDoc vom runlater() sagt 'Run the specified Runnable on the JavaFX Application
Thread at some unspecified time in the future [ \dots ]
und das ist genau was wir brauchen. Lässt man es weg, bekommt man dutzende von Exceptions.

\begin{lstlisting}
  {
Stage stage = new Stage();
	  label = new Label("Hello: " + reminder.toString());
	  Button okButton = new Button("Ok");
	  okButton.setOnAction(e -> {
	      stage.close();
	  });
	  VBox pane = new VBox(10, label, okButton);
	  pane.setAlignment(Pos.CENTER);
	  pane.setPadding(new Insets(10));
	  Scene scene = new Scene(pane);
	  stage.setTitle("Reminder");
	  stage.setScene(scene);
	  stage.setResizable(false);
	  stage.show();
    }
\end{lstlisting}
Der Code in den geschweiften Klammern macht ein simples JavaFX Fenster das den Reminder aufzeigt. Das Label wird mit der reminder.toString() Methode überschrieben. Dem okButton schliesst das Fenster wenn man ihn drückt. Die Komponenten Button und Label tun wir in ein VBox Behälter und machen ihn noch ein bisschen schöner mit setAlignment() und setPadding().


\subsection{Class: MultiReminderNotification}
MultiReminderNotification benutzen wir um alle schon vergangenen Reminder in einem einzigen Fenster darzustellen. Es ist aber auch möglich, eine andere aggrregation von  Remindern mit ihr darzustellen. Der Aufbau der Klasse ist genau gleich wie in JavaFxNotification und ConsoleNotification, nur dass anstatt einem Reminder hat er eine Liste von Reminder.
\begin{lstlisting}
private Collection<Reminder> reminders;

String remindersText = "";
                    int i = 0;
                    for (Reminder r : reminders) {
                        remindersText += "Passed Event No " + ++i + ":\n";
                        remindersText += r.toString() + "\n";
                        System.out.print("added" + r.toString());
                    }
\end{lstlisting}
Hier Iterieren wir durch alle Reminders in der Tabelle und fügen sie zum Label hinzu. Die Reminders sind in der reminders Liste. Damit wir nicht alle Reminders in diesem Popup habem, sonder nur die die bereits vergangen sind oder schon sehr bald erscheinen, werden diese im NotificationHandler noch gefiltert.
Die Methoden dazu wären folgende:
\begin{lstlisting}
criteria.add(new IsPassed());
criteria.add(new IsThisYear());
\end{lstlisting}

\subsection{Class: NotificationHandler}
Der NotificationHandler handelt die Notifications. Dazu werden die einzelnen NotificationTypen mit den passenden CriteriaTesters konfiguriert.
Im Konstruktur wird dem NotificationHandler eine ReminderList übergeben. Für diese Reminders werden beim aufruf der handel() methode die Notifications verwaltet.


in der handle() Methode wird über die ReminderList itteriert. für jeden NotifiationTyp  werden nun die passenden CriteriaTester angegeben.
Wir betrachten nun  lediglich den NotificationTyp, welcher sämtliche Reminders aufpoppen lässt, welche  diesen Monat sind. 

\begin{lstlisting}[caption = NotificationHandler.handle]
Collection<CriteriaTester> importantStuffThisMonth = Arrays.asList(new IsThisMonth());
if (!notifiedReminders.contains(reminder)) {
    /**
    this passes the criteriaTesters to the Reminder itself, and lets the Reminder  send the notification if
    * the criteria are met.
    */
    boolean success = reminder.notifyIf(importantStuffThisMonth);
    if (success) notifiedReminders.add(reminder);
}
\end{lstlisting}
Dann wird für jeden Reminder, welcher noch keine Notification vom entsprechenden Typ abgesendet hat ein Reminder.notfyIf(CriteriaTester) aufgerufen.
Der Reminder testet selbstständig, ob das Kriterium zutifft, und er der Notication den Befehl gibt eine Meldung azusende. Falls dies dies geschieht, meldet der Reminder ein sucess zurück. Der Handler nimmt ihn dann in die Liste der Reminder auf,welche bereits eine Notification abgesednet haben.

Kurz vor dem Datum eines Reminders, wird nochmals auf diesen Reminder hingewiesen.
Der Code funktioniert sehr ähnlich. Es wird aber anstatt ein CriteriaTester.IsThisMonth() ein CriteriaTester.IsNextSecond() übergeben.

\subsubsection{Methode showPastEvents}

Es gibt noch ein dritten Typ von Notifications diese zeigt eine Aggregation der Vergangenen Remindern.Sie werden in einer seperaten Methode abgearbeitet. Der showPastEvents() Methode. Diese löst eine Notification aus, welche mehrere Reminders zusammen darstellt. Desshabl haben wir uns von dem Paradigma gelöst, dass nur der Remidner die Notificaiton auslöst.
Die Methode überprüft auf zwei Kriterien. Erstens ob der Reminder in der Vergangenheit angesiedelt ist  und ob  der Reminder in diesem Jahr war.
Dies geschieht indem man über die ReminderList itteriert. und die passenden Reminders in eine seperate Liste namens passedReminders speichert.
Diese wird dann der MultiReminderNotification übergeben, damit dies die aggregierte Notificaiton absenden kann.

\begin{lstlisting}[caption = NotificationHandler.showPastEvents]
     public void showPastEvents() {
        ArrayList<Reminder> reminderList = reminders.getSerializable();
        ArrayList<Reminder> passedReminders = new ArrayList<>();
        Collection<CriteriaTester> criteria = new ArrayList<>();
        /**
         * the both criteria filter the Reminders for Reminders, which are dated  in th past and dated this year.
         */
        criteria.add(new IsPassed());
        criteria.add(new IsThisYear());

        for (Reminder reminder : reminderList) {
            if (reminder.meetsCriteria(criteria))
                passedReminders.add(reminder);
        }
        // gets sure that a Notification is only sent, if it is not void.
        if (passedReminders.size() != 0) {
            new MultiReminderNotification(passedReminders).send();
        }
    }
\end{lstlisting}

Um sicherzustellen, dass die showPastEvents nur einmal dargestellt weden, muss der Poller sich merken, ob die Methode schon einmal aufgerufen wurde. Falls dies zutrifft, wird auf ein weiterer Aufruf verzichtet.
  \begin{lstlisting}
         public boolean isTrue(Reminder reminder) {
        return reminder.getDate().isAfter(LocalDateTime.now())
                && reminder.getDate().isBefore(LocalDateTime.now().plusSeconds(nextSeconds));
    }
    \end{lstlisting}


subsubsection{IsPassed}
    Diers Filter testet, ob ein Reminder in der Vergangenheit angesiedelt ist.Dazu wird die LocalDateTime.isBefore Methode benutzt um zu testen, ob der Reminder vor dem jetztigen Zeitpunkt stattfindet.
    \begin{lstlisting}
    public boolean isTrue(Reminder reminder) {
        return reminder.getDate().isBefore(LocalDateTime.now());
    }
    \end{lstlisting}
    
     \subsection{IsThisYear}
    Dieser Filter testet ob ein Reminder diesen Jahr stattfindet. Wir vergleichen dabei das Jahr des Reminders mit dem aktuellen Jahr.
    
    \begin{lstlisting}
    public boolean isTrue(Reminder reminder) {
        return reminder.getDate().getYear() == LocalDateTime.now().getYear();
    }
    \end{lstlisting}

    
    \subsection{IsThisMonth}
    Dieser Filter testet ob ein Reminder diesen Monat stattfindet. Dazu  testen wir ob der Reminder im selben Jahr und in demselben Monat sattfindet.
    \begin{lstlisting}
         public boolean isTrue(Reminder reminder) {
        LocalDateTime today = LocalDateTime.now();
        return reminder.getDate().getYear() == today.getYear()
                && reminder.getDate().getMonth() == today.getMonth();
    }
    \end{lstlisting}

    \subsection{IsToday}
    
    Dieser Filter testet ob ein Reminder an diesem Tag stattfindet. Dazu  testen wir ob der Reminder im selben Jahr und in demselben Monat und am selben Tag sattfindet.
    \begin{lstlisting}
         public boolean isTrue(Reminder reminder) {
        LocalDateTime today = LocalDateTime.now();
        return reminder.getDate().getYear() == today.getYear()
                && reminder.getDate().getMonth() == today.getMonth();
    }
    \end{lstlisting}
\section{Resources}
\subsection{Resources}
Die Resourcen die wir gebraucht haben waren:
\begin{itemize}
  \item Java
  \item JavaFX
  \item IntelliJ
  \item Atom
  \item Latex
  \item TornadoFX-Controls Plugin
  \item Git
  \item Github
  \item JUnit

\end{itemize}

\section{Tests}
\subsection{Tests}
Unseren Code haben wir Mittels JUnit getestet. Für das GUI hätte man mit einem Testing Framework arbeiten können, aber davor wurde uns im
Kurs Software Engineering and Design abgeraten.

\section{Tools}
Um die  Versionierung der Dokumentation automatisch generieren zu lassen, haben wir LaTeX so mit Scripts erweitert, so dass die git Head Versionsnummer
direkt ins Dokument eingefügt wird. Somit bleibt diese Information auch auf einem Ausdruck akkurat.



\begin{lstlisting}
#!/bin/sh
OUTPUT="../script/version.tex"

echo "Last compiled: ">$OUTPUT
date >> $OUTPUT

echo "\n">>$OUTPUT

echo "Git HEAD Version: ">> $OUTPUT
git rev-list --count --first-parent HEAD >>$OUTPUT
\end{lstlisting}
Dann wird ein cleanup durchgeführt
\begin{lstlisting}
#!/bin/sh
OUTPUT="../script/version.tex"
echo "Fetching version information failed. Please enable shell-escape in your \LaTeX \~  compiler.">$OUTPUT
\end{lstlisting}

\section{Versionskontrolle}
Manuelle Version: 1.0.0
\\

\noindent
Automatische Versionierung:
\immediate\write18{../script/versionInfo.sh}
Fetching version information failed. Please enable shell-escape in your \LaTeX \~  compiler.

\immediate\write18{../script/cleanup.sh}







\end{document}
