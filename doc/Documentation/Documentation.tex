%\documentclass[a4paper,10pt]{report}
\documentclass[11pt,titelpage]{scrartcl}
\usepackage[utf8]{inputenc}
\usepackage[ngerman]{babel}
\usepackage{graphicx}
\usepackage{fancyhdr}
\usepackage{fancyref}
\usepackage{lscape}
\usepackage{listings}



%must be before gloassary stuff\usepackage{hyperref}


\makeglossaries



% Title Page
\title{Alarm Clock Dokumentation}
\author{Jonathan Hyams \\Pascal Schmalz}
\titlehead{\centering\includegraphics[width=6cm]{../Requirements/img/clock.png}}

%Make the Header
\makeatletter
\let\runauthor\@author
\let\runtitle\@title
\makeatother
\rhead{\runauthor}
\chead{\runtitle}
\lhead{\begin{picture}(0,0) \put(0,0){\includegraphics[scale=0.5]{../Requirements/img/bfh.png}} \end{picture}}



\begin{document}

\thispagestyle{empty}
\maketitle
\pagebreak
\tableofcontents

\pagestyle{fancy}


\begin{abstract}
\end{abstract}
\pagebreak

\section{Zweck des Dokument}
Dieses Dokument soll dazu dienen, den Code des Projektes zu erklären für Leute, die sich damit auseinander setzen wollen.
%makes sure the whole glossary gets printed even when acronyms are not defined

\section{Main.java}
Dies ist die Klasse die die Main methode enthält. Sie erbt von der Klasse Application, da Sie das JavaFX-GUI launched.
\begin{lstlisting}
   Parent root = FXMLLoader.load(getClass().getResource(windowName));
\end{lstlisting}
windowName ist der Name der fxml Datei (mainWindow.fxml), in der das Aussehen und der Controller definiert wird. "mainWindow.fxml" befindet sich im package "resources".







\end{document}
